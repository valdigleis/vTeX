%---------------------------------------------------------------------------------------
%	Configuração para estilo matemático
%---------------------------------------------------------------------------------------
\everymath{\displaystyle}

%---------------------------------------------------------------------------------------
%	Configuração do pacote Epigraph
%---------------------------------------------------------------------------------------
\setlength\epigraphrule{1pt}
%\renewcommand\textflush{flushright}
\renewcommand\epigraphsize{\normalsize\itshape}

%---------------------------------------------------------------------------------------
%	Configuração do pacote Hyperref
%---------------------------------------------------------------------------------------
\hypersetup{
	% show Acrobat’s toolbar?
	pdftoolbar=true, 
	% show Acrobat’s menu?       
	pdfmenubar=true, 
	% window fit to page when opened       
	pdffitwindow=false,
	% fits the width of the page to the window   
	pdfstartview={FitH},
	% title
	pdftitle={titulo},
	 % author
	pdfauthor={Linus van Pelt}, 
	% list of keywords
	pdfkeywords={Lógica, Computação}, 
	% links in new PDF window
	pdfnewwindow=true, 
	 % false: boxed links; true: colored links
	colorlinks=true, 
	% color of internal links      
	linkcolor=maroon,
	% color of links to bibliography       
	citecolor=teal,
	% color of file links     
	filecolor=filelink,    
	% color of external links     
	urlcolor=externallink
}

%---------------------------------------------------------------------------------------
%	Configuração a geometria do documento
%---------------------------------------------------------------------------------------
\geometry{
	paper=a4paper,
	inner=3cm,
	outer=5cm,
	top=2.5cm,
	bottom=3cm,
	headheight=15pt, % Header height
	%footskip=1.5cm, % Space from the bottom margin to the baseline of the footer
	headsep=10pt,
	marginparwidth=4cm,
	marginparsep=0.5cm
}


% Contador para os rodapés
\newcounter{sidefootnote}

% Faz o contador ser reiniciado a cada capítulo
\makeatletter
\@addtoreset{sidefootnote}{chapter}
\makeatother

%---------------------------------------------------------------------------------------
%	Configuração do espaçamento
%---------------------------------------------------------------------------------------
\renewcommand{\baselinestretch}{1} 

%---------------------------------------------------------------------------------------
%	Configuração do ambiente para códigos C usado no pacote listings
%---------------------------------------------------------------------------------------
\lstdefinestyle{Cstyle}{
	backgroundcolor=\color{black!5},   
	commentstyle=\color{magenta},
	keywordstyle=\color{blue},
	numberstyle=\tiny\color{black},
	stringstyle=\color{purple},
	basicstyle=\ttfamily\footnotesize,
	breakatwhitespace=false,         
	breaklines=true,                 
	captionpos=b,                    
	keepspaces=false,                 
	numbers=left,                    
	numbersep=5pt,                  
	showspaces=false,                
	showstringspaces=false,
	showtabs=false,                  
	tabsize=1,
	language=C
}

%---------------------------------------------------------------------------------------
%	Configuração do ambiente para códigos Haskell usado no pacote listings
%---------------------------------------------------------------------------------------
\lstdefinestyle{Hstyle}{
	backgroundcolor=\color{black!5},   
	commentstyle=\color{purple},
  keywordstyle=\color{cyan},
	numberstyle=\tiny\color{black},
	stringstyle=\color{teal},
	basicstyle=\ttfamily\footnotesize,
	breakatwhitespace=false,         
	breaklines=true,                 
	captionpos=b,                    
	keepspaces=false,                 
	numbers=left,                    
	numbersep=5pt,                  
	showspaces=false,                
	showstringspaces=false,
	showtabs=false,                  
	tabsize=1,
	language=Haskell
}